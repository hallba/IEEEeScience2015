\documentclass[conference]{IEEEtran}

\usepackage[british]{babel}
\usepackage{cite}
\usepackage{graphicx}
\usepackage[hyphens]{url}
%\usepackage[pdftex]{hyperref}


% correct bad hyphenation here
%\hyphenation{op-tical net-works semi-conduc-tor}


\begin{document}
%
% paper title
\title{Reproducibility as a Technical Specification}


% author names and affiliations
% use a multiple column layout for up to three different
% affiliations
\author{\IEEEauthorblockN{Tom Crick}
\IEEEauthorblockA{Department of Computing \& Information Systems\\
Cardiff Metropolitan University, UK\\
Email: {\url{tcrick@cardiffmet.ac.uk}}}
\and
\IEEEauthorblockN{Benjamin A. Hall}
\IEEEauthorblockA{MRC Cancer Unit\\
University of Cambridge, UK\\
Email: {\url{bh418@mrc-cu.cam.ac.uk}}}
\and
\IEEEauthorblockN{Samin Ishtiaq}
\IEEEauthorblockA{Microsoft Research\\
Cambridge, UK\\
Email: {\url{samin.ishtiaq@microsoft.com}}}}

% conference papers do not typically use \thanks and this command
% is locked out in conference mode. If really needed, such as for
% the acknowledgment of grants, issue a \IEEEoverridecommandlockouts
% after \documentclass


% use for special paper notices
%\IEEEspecialpapernotice{(Invited Paper)}


% make the title area
\maketitle


\begin{abstract}
Reproducibility of scientific discoveries should be a certainty. As
the product of several person-years' worth of effort, results --
whether disseminated through academic journals, conferences or
commercial ventures -- should generally be expected to be at
repeatable by other researchers. On the surface this appears trivial,
but a variety of factors can stand in the way. Whilst there has been
detailed cross-disciplinary discussions of the various social,
cultural and ideological drivers and (potential) solutions, one factor
which has had less focus is the concept of reproducibility as a
\emph{technical} challenge, with the potential for unambiguously
defined standards and specifications.

In this paper, we discuss such a standard by defining the features
which constitute a specification, and present cyberinfrastructure and
associated workflow for a service which would enable such a
specification to be verified and validated. In addition to addressing
a pressing need for the scientific community, we further speculate on
the potential contribution to the wider software development community
of services which automate \emph{de novo} compilation and testing of
code from source.
\end{abstract}

% For peer review papers, you can put extra information on the cover
% page as needed:
% \ifCLASSOPTIONpeerreview
% \begin{center} \bfseries Keywords here... \end{center}
% \fi
%
% For peerreview papers, this IEEEtran command inserts a page break and
% creates the second title. It will be ignored for other modes.
%\IEEEpeerreviewmaketitle

\begin{IEEEkeywords}
Reproducibility, Artifact evaluation, Cyberinfrastructure, Research
software, Benchmarks, Verification, Validation
\end{IEEEkeywords}


%\newpage

% trigger a \newpage just before the given reference
% number - used to balance the columns on the last page
% adjust value as needed - may need to be readjusted if
% the document is modified later
%\IEEEtriggeratref{28}
% The "triggered" command can be changed if desired:
%\IEEEtriggercmd{\enlargethispage{-5in}}

% references section
\bibliographystyle{IEEEtran}
\bibliography{ieee2015}

% that's all folks
\end{document}


