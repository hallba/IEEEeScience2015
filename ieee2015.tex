\documentclass{llncs}
%
\usepackage{makeidx}  % allows for indexgeneration
\usepackage[british]{babel}
\usepackage{url}
\usepackage[pdftex,colorlinks=true]{hyperref}
\usepackage{graphicx}
%\fussy


\title{Reproducibility as a technical specification}

\author{Tom Crick\inst{1} \and Benjamin A. Hall\inst{2} \and Samin Ishtiaq\inst{3}}

\institute{Department of Computing \& Information Systems\\Cardiff Metropolitan University, UK\\
\email{tcrick@cardiffmet.ac.uk}
\and
MRC Cancer Unit, University of Cambridge, UK\\
\email{bh418@mrc-cu.cam.ac.uk}
\and
Microsoft Research Cambridge, UK\\
\email{samin.ishtiaq@microsoft.com}
}

\raggedbottom
\begin{document}
%
\frontmatter          % for the preliminaries
%
\pagestyle{headings}  % switches on printing of running heads
%\addtocmark{} % additional mark in the TOC

\maketitle

\begin{abstract}

Reproducibility of scientific discoveries should be a certainty. As the product 
of several person-years worth of effort results, whether disseminated through
academic journals, conferences or commercial ventures, can be reasonably expected
to be at a bare minimum repeatable. On the surface this appears trivial, but a 
variety of factors can stand in the way. Whilst there has been detailed cross-
displinary discussion of the political and ideological drivers and solutions, 
one factor which has been less discussed is the concept of reproduciblity as 
a \emph{technical} challenge, with the potential for unambiguously defined 
standards and specifications. Here we discuss such a standard by defining the
features which constitute a specification, and discuss a hypothetical cloud
based service which would enable such a specification to be tested. In addition
to serving a specific need for the scientific community, we further speculate on the 
wider importance to the broader software development community of services which
automate \emph{de novo} compilation and testing of code from source.

\end{abstract}

% Keywords for Easychair:
% Reproducibility
% Artifact evaluation
% Benchmarks
% Verification
% Validation


\section{Introduction}\label{intro}


\bibliographystyle{splncs}
\bibliography{ieee2015}

\end{document}
